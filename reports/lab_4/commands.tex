\newcommand{\threeimage}[4]{
\begin{figure}[h!]  
    \centering 
    \subfigure[]{
        \includegraphics[width=0.3\linewidth, height=0.25\linewidth]{#1} 
        \label{fig:f_ #1} }  
        %\hspace{4ex}
    \subfigure[]{
    \includegraphics[width=0.3\linewidth, height=0.25\linewidth]{#2} 
    \label{fig:f_ #2} }
   % \hspace{4ex}
    \subfigure[]{ 
        \includegraphics[width=0.3\linewidth, height=0.25\linewidth]{#3} 
        \label{fig:f_ #3} }  
    \caption{Выборки из #4 объемом: 
    \subref{fig:f_ #1} 20; 
    \subref{fig:f_ #2} 60; 
    \subref{fig:f_ #3} 100} 
    \label{fig:f_ #1#2#3}
\end{figure}}

\newcommand{\triplethreeimage}[4]{
\begin{figure}[h!]  
    \centering 
    \subfigure[]{
        \includegraphics[width=.83\linewidth, height=.3\linewidth]{#1} 
        \label{fig:f_ #1} }\\  
    %\hspace{4ex}
    \subfigure[]{
        \includegraphics[width=.83\linewidth, height=.3\linewidth]{#2} 
        \label{fig:f_ #2} }\\
    %\hspace{4ex}
    \subfigure[]{ 
        \includegraphics[width=.83\linewidth, height=.3\linewidth]{#2} 
        \label{fig:f_ #3} }  
    \caption{Для выборок из #4 объемом: 
    \subref{fig:f_ #1} 20; 
    \subref{fig:f_ #2} 60; 
    \subref{fig:f_ #3} 100} 
    \label{fig:f_ #1#2#3}
\end{figure}}