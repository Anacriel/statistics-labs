% !TEX options=--shell-escape
\documentclass[12pt]{article}
\usepackage[utf8]{inputenc}
\usepackage{lipsum}
\usepackage{afterpage}
\usepackage{mathtools}
\usepackage{xcolor}
\usepackage[12pt]{extsizes}
\usepackage[english,russian]{babel}
\usepackage{cite}
\usepackage{minted}
\usepackage{amsmath, esint, setspace, fancyhdr, amsfonts, bookmark, blindtext}
\usepackage{graphicx}
\usepackage{subfigure}
\usepackage{titlesec}

\graphicspath{{Figures/}}
\DeclareGraphicsExtensions{.pdf,.png,.jpg}
\usemintedstyle{tango}
\definecolor{dhscodebg}{rgb}{0.95,0.95,0.95}


\setlength{\textheight}{8in}
\setlength{\textwidth}{6.6in}
\setlength{\headheight}{0in}
\setlength{\headsep}{0.2in}
\setlength{\topmargin}{0in}
\setlength{\oddsidemargin}{0in}
\setlength{\evensidemargin}{0in}
\setlength{\parindent}{.3in}

\doublespacing
\renewcommand{\baselinestretch}{1.4} 

\begin{document}
\begin{titlepage}

\begin{center}
Санкт-Петербургский политехнический университет Петра Великого\\
Институт прикладной математики и механики\\
Кафедра прикладной математики\\
\end{center}


\vspace{2.5cm}

\begin{center}
{\large {\bfseries ОТЧЕТ}}\\

\bigskip \bfseries{Тема:} {\bfseries \emph{Оценки законов распределений}}
\end{center}

\vspace{1.5cm}

\begin{flushleft}
Направление: 01.03.02 Прикладная математика и информатика

\vspace{1.5cm}

Выполнил студент гр. 33631/4 \hfill{Камалетдинова Ю.} \\ 

\vspace{0.5cm} Преподаватель \hfill{Баженов А.}
\vspace{1cm}

\end{flushleft}

\vspace{2.7cm}

\begin{center}
Санкт-Петербург\\
2019
\end{center}

\end{titlepage}



\tableofcontents
\addtocontents{toc}{~\hfill\par}
\vfill ~
\setcounter{section}{0}


%%%%%%%%%%%%%%%%%%%%%%%%%%%%%%%%%%%%%%%%%%

\newpage 
\section*{Постановка задачи}
\addcontentsline{toc}{section}{Постановка задачи}
\indent{\indentВ данной лабораторной работе требуется вычислить некоторые из характеристик положения $N = 1000$ раз для выборок объемами $n = 20, 50, 100$ и проанализировать полученные результаты. Также необходимо установить, в каком соотношении находятся вычисленные характеристики для каждого из распределений, приведенных ниже}

\begin{equation}
	\label{dist:1}
	N(x, 0, 1) = \frac{1}{\sqrt{2\pi}}e^{-\frac{x^2}{2}} \;\text{-- стандартное нормальное}
\end{equation}
 
\begin{equation}
	C(x, 0, 1) = \frac{1}{\pi(1 + x^2)} \;\text{-- Коши} \label{dist:2}
\end{equation}

\begin{equation}
	L(x, 0, \frac{1}{\sqrt{2}}) = \frac{1}{\sqrt{2}}e^{{-\sqrt{2}|x|}} \;\text{-- Лаплас} \label{dist:3}
\end{equation}

\begin{equation} 
	U(x, -\sqrt{3}, \sqrt{3}) = 
    \begin{cases}
        \frac{1}{2\sqrt{3}}, \: |x| \leq \sqrt{3}\\
        \;\; 0, \:\:\:|x| > \sqrt{3}
    \end{cases}
    \;\text{-- равномерное} \label{dist:4}
\end{equation}

\begin{equation}
    P(\lambda) = \frac{e^{-\lambda}}{k!}\lambda^k \;\text{-- Пуассон} \label{dist:5}
\end{equation}

\indent{Приведем формулы для вычисления характеристик положения}

\begin{equation}
	\label{char:1}
	\overline{x} = \overline{x_n} = \frac{1}{n} \sum_{i=1}^{n}{x_i} \;\text{-- выборочное среднее}
\end{equation}
\begin{equation} 
	med \; x = 
    \begin{cases}
        \;\;\; x_{(k+1)}, \:\;\;\; n = 2k + 1 \\
        \frac{x_{(k)} + x_{(k+1)}}{2}, \; n = 2k
    \end{cases}
    \;\text{-- медиана} \label{char:2}
\end{equation}

\begin{equation}
	\label{char:3}
	z_R = \frac{x_{(1)} + x_{(n)}}{2}  \;\text{-- полусумма экстремальных значений}
\end{equation}

\begin{equation}
	\label{char:4}
	z_Q = \frac{Q_1 + Q_3}{2}  \;\text{-- полусумма квартилей}
\end{equation}

\begin{equation}
	\label{char:5}
	z_{tr} = \frac{1}{n - 2r} \sum_{i = r + 1}^{n - r}{x_{(i)}}  \;\text{-- усеченное среднее}
\end{equation}

\indent{\underline{Замечание:} $r$ -- число наблюдений, оставшихся после усечения в характеристике \eqref{char:5}, $r = \alpha n$, где $\alpha$, как правило, равняется $0.1$. В таком случае мы не вовлекаем $10$\% наибольших и $10$\% наименьших значений в вычисление усеченного среднего.}

%%%%%%%%%%%%%%%%%%%%%%%%%%%%%%%%%%%%%%%%%%

\section*{Реализация метода}
\addcontentsline{toc}{section}{Реализация}
\indent{\indentДля выполнения поставленной задачи будем пользоваться библиотеками для языка Python: \textit{numpy, scipy} -- расчеты, законы распределения вероятностей; \textit{matplotlib, seaborn} -- визуализация результатов. Ход работы:}
\begin{itemize}
    \item Задаем распределение с заданными параметрами 
    \item Генерирем случайные выборки из распределений размерами $n = 20, 50, 100$
    \item Для каждого из распределений вычисляем характеристики положения $N = 1000$ раз
    \item Вычисляем математическое ожидание и дисперсию для каждой вычисленной характеристики по формулам:
    \begin{equation}
    \label{moment:1}
      E(z) = \overline{z} = \frac{1}{N}\sum_{i=1}^{N}{z_i}
    \end{equation}
    \begin{equation}
    \label{moment:2}
      D(z) = \overline{z^2} - (\overline{z})^2
    \end{equation}
\end{itemize}

%%%%%%%%%%%%%%%%%%%%%%%%%%%%%%%%%%%%%%%%%%

\newpage
\section*{Результат}
\addcontentsline{toc}{section}{Результат}

\textbf{Нормальное распределение} с параметрами 0, 1\\
\\
\begin{tabular}{ | c | c | c | c | c | c |}
\hline
$n = 20$ & $\overline{x}$ & $med\;x$ & $z_R$ & $z_Q$ & $z_{tr}$ \\ \hline
$E(z)$ & 0.0041 & 0.0053 & 0.0005 & 0.0046 & 0.0054 \\ \hline
$D(z)$ & 0.0511 & 0.0753 & 0.1405 & 0.0586 & 0.0542 \\ \hline
\end{tabular}
\\
\\ \\
\begin{tabular}{ | c | c | c | c | c | c |}
\hline
$n = 50$ & $\overline{x}$ & $med\;x$ & $z_R$ & $z_Q$ & $z_{tr}$ \\ \hline
$E(z)$ & 0.0013 & -0.0004 & 0.0170 & 0.0003 & -0.0010 \\ \hline
$D(z)$ & 0.0211 & 0.0313 & 0.1146 & 0.0251 & 0.0222 \\ \hline
\end{tabular}
\\ 
\\ \\
\begin{tabular}{ | c | c | c | c | c | c |}
\hline
$n = 100$ & $\overline{x}$ & $med\;x$ & $z_R$ & $z_Q$ & $z_{tr}$ \\ \hline
$E(z)$ & 0.0028 & 0.0026 & 0.0037 & 0.0034 & 0.0023 \\ \hline
$D(z)$ & 0.0100 & 0.0152 & 0.0887 & 0.0122 & 0.0106 \\ \hline
\end{tabular}
\\ \\
\indent{Соотношение дисперсий при $n = 100$: $\;\overline{x} < z_{tr} < z_Q < med\;x < z_R$} 
\\ \\
\textbf{Равномерное распределение} на отрезке $[-\sqrt{3}, \sqrt{3}]$
\\  \\
\begin{tabular}{ | c | c | c | c | c | c |}
\hline
$n = 20$ & $\overline{x}$ & $med\;x$ & $z_R$ & $z_Q$ & $z_{tr}$ \\ \hline
$E(z)$ & -0.0068 & -0.0119 & 0.0011 & -0.0113 & -0.0082 \\ \hline
$D(z)$ & 0.0554 & 0.1404 & 0.0141 & 0.0728 & 0.0742 \\ \hline
\end{tabular}
\\
\\ \\ 
\begin{tabular}{ | c | c | c | c | c | c |}
\hline
$n = 50$ & $\overline{x}$ & $med\;x$ & $z_R$ & $z_Q$ & $z_{tr}$ \\ \hline
$E(z)$ & 0.0032 & 0.0049 & -0.0004 & 0.0035 & 0.0040 \\ \hline
$D(z)$ & 0.0209 & 0.0582 & 0.0022 & 0.0304 & 0.0288 \\ \hline
\end{tabular}
\\
\\ \\ 
\begin{tabular}{ | c | c | c | c | c | c |}
\hline
$n = 100$ & $\overline{x}$ & $med\;x$ & $z_R$ & $z_Q$ & $z_{tr}$ \\ \hline
$E(z)$ & 0.0015 & 0.0015 & 0.0011 & 0.0017 & 0.0019 \\ \hline
$D(z)$ & 0.0098 & 0.0292 & 0.0007 & 0.0156 & 0.0142 \\ \hline
\end{tabular}
\\ \\ 
\indent{Соотношение дисперсий при $n = 100$: $\;z_R < \overline{x} < z_{tr} < z_Q < med\;x$}
\\ \\
\textbf{Распределение Коши} с параметрами 0, 1
\\  \\
\begin{tabular}{ | c | c | c | c | c | c |}
\hline
$n = 20$ & $\overline{x}$ & $med\;x$ & $z_R$ & $z_Q$ & $z_{tr}$ \\ \hline
$E(z)$ & -4.1847 & -0.0185 & -41.9652 & -0.0013 & 0.0043 \\ \hline
$D(z)$ & 9829.5729 & 0.1450 & 982309.3260 & 0.3753 & 0.4334 \\ \hline
\end{tabular}
\\
\\ \\ 
\begin{tabular}{ | c | c | c | c | c | c |}
\hline
$n = 50$ & $\overline{x}$ & $med\;x$ & $z_R$ & $z_Q$ & $z_{tr}$ \\ \hline
$E(z)$ & 3.7413 & 0.0051 & 95.1768 & -0.0011 & -0.0022 \\ \hline
$D(z)$ & 14981.2878 & 0.0483 & 9324487.2831 & 0.1102 & 0.1094 \\ \hline
\end{tabular}
\\
\\ \\ 
\begin{tabular}{ | c | c | c | c | c | c |}
\hline
$n = 100$ & $\overline{x}$ & $med\;x$ & $z_R$ & $z_Q$ & $z_{tr}$ \\ \hline
$E(z)$ & 2.6928 & -0.0082 & 133.6930 & -0.0080 & -0.0116 \\ \hline
$D(z)$ & 4451.3761 & 0.0247 & 11101512.6370 & 0.0522 & 0.0513 \\ \hline
\end{tabular}
\\ \\
\indent{Соотношение дисперсий при $n = 100$: $\;med\;x < z_{tr} < z_Q < \overline{x} < z_R$}
\\ \\
\textbf{Распределение Лапласа} с параметрами 0, $\frac{1}{\sqrt{2}}$
\\ \\ 
\begin{tabular}{ | c | c | c | c | c | c |}
\hline
$n = 20$ & $\overline{x}$ & $med\;x$ & $z_R$ & $z_Q$ & $z_{tr}$ \\ \hline
$E(z)$ & -0.0113 & -0.0035 & -0.0344 & -0.0087 & -0.0070 \\ \hline
$D(z)$ & 0.0550 & 0.0348 & 0.4406 & 0.0519 & 0.0427 \\ \hline
\end{tabular}
\\
\\ \\ 
\begin{tabular}{ | c | c | c | c | c | c |}
\hline
$n = 50$ & $\overline{x}$ & $med\;x$ & $z_R$ & $z_Q$ & $z_{tr}$ \\ \hline
$E(z)$ & 0.0028 & -0.0010 & 0.0085 & 0.0035 & 0.0019 \\ \hline
$D(z)$ & 0.0193 & 0.0130 & 0.3839 & 0.0206 & 0.0156 \\ \hline
\end{tabular}
\\
\\ \\ 
\begin{tabular}{ | c | c | c | c | c | c |}
\hline
$n = 100$ & $\overline{x}$ & $med\;x$ & $z_R$ & $z_Q$ & $z_{tr}$ \\ \hline
$E(z)$ & -0.0059 & -0.0029 & -0.0104 & -0.0047 & -0.0051 \\ \hline
$D(z)$ & 0.0097 & 0.0056 & 0.3931 & 0.0093 & 0.0071 \\ \hline
\end{tabular}
\\ \\
\indent{Соотношение дисперсий при $n = 100$: $\;med\;x < z_{tr} < z_Q < \overline{x} < z_R$}
\\ \\
\textbf{Распределение Пуассона} с параметром $\lambda = 7$ \\
\\ 
\begin{tabular}{ | c | c | c | c | c | c |}
\hline
$n = 20$ & $\overline{x}$ & $med\;x$ & $z_R$ & $z_Q$ & $z_{tr}$ \\ \hline
$E(z)$ & 6.9844 & 6.8310 & 7.4820 & 6.9067 & 6.8978 \\ \hline
$D(z)$ & 0.3285 & 0.5259 & 1.0587 & 0.4049 & 0.3441 \\ \hline
\end{tabular}
\\
\\ \\ 
\begin{tabular}{ | c | c | c | c | c | c |}
\hline
$n = 50$ & $\overline{x}$ & $med\;x$ & $z_R$ & $z_Q$ & $z_{tr}$ \\ \hline
$E(z)$ & 7.0165 & 6.8640 & 7.7515 & 6.9400 & 6.9287 \\ \hline
$D(z)$ & 0.1561 & 0.2665 & 0.8315 & 0.2299 & 0.1639 \\ \hline
\end{tabular}
\\
\\ \\ 
\begin{tabular}{ | c | c | c | c | c | c |}
\hline
$n = 100$ & $\overline{x}$ & $med\;x$ & $z_R$ & $z_Q$ & $z_{tr}$ \\ \hline
$E(z)$ & 7.0063 & 6.8755 & 7.9230 & 6.9088 & 6.9181 \\ \hline
$D(z)$ & 0.0714 & 0.1482 & 0.7266 & 0.1074 & 0.0746 \\ \hline
\end{tabular}
\\ \\
\indent{Соотношение дисперсий при $n = 100$: $ \overline{x} < z_{tr} < z_Q < med\;x < z_R$}
\\ \\
\indent{Обратившись к полученным соотношениям дисперсий выборок из распределений, можно сделать вывод о том, что полусумма экстремальных значений $z_R$ имеет наибольший разброс относительно математического ожидания. Данное суждение не сходится с результатами для случая равномерного распределения на отрезке. Медиана равномерного распределения на $[a, b]$ есть $med = \frac{a+b}{2}$, что и является полусуммой значений в крайних точках отрезка.}\\
\indent{Также полученные результаты говорят о том, что в в случае симметричного распределения, исходя из дисперсий, наиболее выгодно использовать выборочное среднее, чем медиану, хотя они и оценивают одну и ту же величину в данном случае. Но для распределения Лапласа медиана становится более эффективной.}\\
\indent{Дисперсии таких характеристик, как усеченное среднее или полусумма квартилей, показывают среднее отклонение относительно остальных. Усеченное среднее представляет собой некий баланс между медианой и выборочным средним (является ими в частных случаях параметра $\alpha$)}

%%%%%%%%%%%%%%%%%%%%%%%%%%%%%%%%%%%%%%%%%%

\newpage
\begin{thebibliography}{}
	\bibitem{ms_1} \textit{Кадырова Н. О.} Теория вероятностей и математическая статистика. Статистический анализ данных: учеб. пособие / \textit{Н. О. Кадырова, Л. В. Павлова, И. Е. Ануфриев.} - СПб.: Изд-во Политехн. ун-та, 2010. -54с.
\end{thebibliography}


\end{document}{}